\begin{zed}
    [ALUMNO] \\
    GRADO == \nat \\
    ESTADO ::= inscripto | promueve | repite \\
    REP ::= AlumnoEsGraduado | alumnoNoEsGraduado | alumnoNoEncontrado
\end{zed}

\begin{schema}{MisEstudiantes}
    registrados: \power ALUMNO \\
    inscripciones: ALUMNO \pfun (GRADO \cross ESTADO)
\end{schema}

\begin{schema}{InscripcionesInv}
    MisEstudiantes
    \where
    \dom inscripciones = registrados
\end{schema}

\begin{schema}{MaximoGradoInv}
    MisEstudiantes
    \where
    \forall a : registrados @ (inscripciones \; a).1 \leq 12
\end{schema}

\begin{schema}{MisEstudiantesInicial}
    MisEstudiantes
    \where
    registrados = \emptyset \\
    inscripciones = \emptyset
\end{schema}

\begin{schema}{InscribirAlumnoOk}
    \Delta MisEstudiantes \\
    alumno? : ALUMNO
    \where
    alumno? \notin registrados \\
    inscripciones' = inscripciones \cup \{alumno? \mapsto (1, inscripto) \} \\
    registrados' = registrados \cup \{alumno?\}
\end{schema}

\begin{schema}{InscribirAlumnoRegistradoE}
    \Xi MisEstudiantes \\
    alumno? : ALUMNO
    \where
    alumno? \in registrados
\end{schema}

\begin{zed}
    InscribirAlumno == InscribirAlumnoOk \lor InscribirAlumnoRegistradoE
\end{zed}

\begin{schema}{ReinscribirAlumnoPromovidoOk}
    \Delta MisEstudiantes \\
    alumno? : ALUMNO
    \where
    alumno? \in registrados \\
    (inscripciones \; alumno?).1 < 12 \\ 
    (inscripciones \; alumno?).2 = promueve \\
    inscripciones' = inscripciones \oplus \{alumno? \mapsto ((inscripciones \; alumno?).1 + 1, inscripto) \} \\
    registrados' = registrados
\end{schema}

\begin{schema}{ReinscribirAlumnoRepitenteOk}
    \Delta MisEstudiantes \\
    alumno? : ALUMNO
    \where
    alumno? \in registrados \\
    (inscripciones \; alumno?).1 \leq 12 \\ 
    (inscripciones \; alumno?).2 = repite \\
    inscripciones' = inscripciones \oplus \{alumno? \mapsto ((inscripciones \; alumno?).1, inscripto) \} \\
    registrados' = registrados
\end{schema}

\begin{schema}{ReinscribirAlumnoNoEncontradoE}
    \Xi MisEstudiantes \\
    alumno? : ALUMNO
    \where
    alumno? \notin registrados
\end{schema}

\begin{schema}{ReinscribirAlumnoGraduadoE}
    \Xi MisEstudiantes \\
    alumno? : ALUMNO
    \where
    alumno? \in registrados \\
    ((inscripciones \; alumno?)).1 = 12 \\
    ((inscripciones \; alumno?)).2 = promueve 
\end{schema}

\begin{schema}{ReinscribirAlumnoDobleInscripE}
    \Xi MisEstudiantes \\
    alumno? : ALUMNO
    \where
    alumno? \in registrados \\
    (inscripciones \; alumno?).2 = inscripto
\end{schema}

\begin{zed}
    ReinscribirAlumnoE == ReinscribirAlumnoGraduadoE \lor ReinscribirAlumnoDobleInscripE \\
                             \lor ReinscribirAlumnoNoEncontradoE \\
    ReinscribirAlumnoOk == ReinscribirAlumnoPromovidoOk \lor ReinscribirAlumnoRepitenteOk \\
    ReinscribirAlumno == ReinscribirAlumnoOk \lor ReinscribirAlumnoE
\end{zed}

\begin{schema}{CerrarInscripcionOk}
    \Delta MisEstudiantes \\
    alumno?: ALUMNO \\
    estado?: ESTADO
    \where
    alumno? \in registrados \\
    inscripciones' = inscripciones \oplus \{alumno? \mapsto ((inscripciones \; alumno?).1, estado?) \} \\
    registrados' = registrados
\end{schema}

\begin{schema}{CerrarInscripcionEstadoInvalidoE}
    \Xi MisEstudiantes \\
    estado?: ESTADO
    \where
    estado? = inscripto
\end{schema}

\begin{schema}{CerrarInscripcionAlumnoNoEncontradoE}
    \Xi MisEstudiantes \\
    alumno? : ALUMNO
    \where
    alumno? \notin registrados
\end{schema}

\begin{zed}
    CerrarInscripcionE == CerrarInscripcionEstadoInvalidoE \lor CerrarInscripcionAlumnoNoEncontradoE \\
    CerrarInscripcion == CerrarInscripcionOk \lor CerrarInscripcionE
\end{zed}

\begin{schema}{AlumnoEsGraduadoSiOk}
    \Xi MisEstudiantes \\
    alumno? : ALUMNO \\
    rep!: REP
    \where
    alumno? \in registrados \\
    (inscripciones \; alumno?).1 = 12 \\
    (inscripciones \; alumno?).2 = promueve \\
    rep! = alumnoEsGraduado
\end{schema}

\begin{schema}{AlumnoEsGraduadoNoOk}
    \Xi MisEstudiantes \\
    alumno? : ALUMNO \\
    rep!: REP
    \where
    alumno? \in registrados \\
    (inscripciones \; alumno?).1 \neq 12 \lor (inscripciones \; alumno?).2 \neq promueve \\
    rep! = alumnoNoEsGraduado
\end{schema}

\begin{schema}{AlumnoEsGraduadoNoEncontradoE}
    \Xi MisEstudiantes \\
    alumno? : ALUMNO \\
    rep!: REP
    \where
    alumno? \notin registrados \\
    rep! = alumnoNoEncontrado
\end{schema}

\begin{zed}
    AlumnoEsGraduadoOk == AlumnoEsGraduadoSiOk \lor AlumnoEsGraduadoNoOk \\
    AlumnoEsGraduado == AlumnoEsGraduadoOk \lor AlumnoEsGraduadoNoEncontradoE \\
\end{zed}

