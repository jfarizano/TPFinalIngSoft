\begin{zed}
    [ALUMNO]
\end{zed}
\begin{zed}
    GRADO == \nat
\end{zed}
\begin{zed}
    ESTADO ::= inscripto | promueve | repite
\end{zed}
\begin{zed}
    REP ::= alumnoEsGraduado | alumnoNoEsGraduado | alumnoNoEncontrado
\end{zed}

\begin{schema}{MisEstudiantes}
    registrados: \power ALUMNO \\
    inscripciones: ALUMNO \pfun (GRADO \cross ESTADO)
\end{schema}

\begin{schema}{InscripcionesInv}
    MisEstudiantes
    \where
    \dom inscripciones = registrados
\end{schema}

\begin{schema}{MaximoGradoInv}
    MisEstudiantes
    \where
    \forall a : registrados @ (inscripciones a).1 \leq 12
\end{schema}

\begin{schema}{MisEstudiantesInicial}
    MisEstudiantes
    \where
    registrados = \emptyset \\
    inscripciones = \emptyset
\end{schema}

\begin{schema}{InscribirAlumnoOk}
    \Delta MisEstudiantes \\
    alumno? : ALUMNO
    \where
    alumno? \notin registrados \\
    inscripciones' = inscripciones \cup \{alumno? \mapsto (1, inscripto) \} \\
    registrados' = registrados \cup \{alumno?\}
\end{schema}

\begin{schema}{InscribirAlumnoRegistradoE}
    \Xi MisEstudiantes \\
    alumno? : ALUMNO
    \where
    alumno? \in registrados
\end{schema}

\begin{zed}
    InscribirAlumno \defs InscribirAlumnoOk \lor InscribirAlumnoRegistradoE
\end{zed}

\begin{schema}{ReinscribirAlumnoPromovidoOk}
    \Delta MisEstudiantes \\
    alumno? : ALUMNO
    \where
    alumno? \in registrados \\
    (inscripciones alumno?).1 < 12 \\ 
    (inscripciones alumno?).2 = promueve \\
    inscripciones' = inscripciones \oplus \{alumno? \mapsto ((inscripciones alumno?).1 + 1, inscripto) \} \\
    registrados' = registrados
\end{schema}

\begin{schema}{ReinscribirAlumnoRepitenteOk}
    \Delta MisEstudiantes \\
    alumno? : ALUMNO
    \where
    alumno? \in registrados \\
    (inscripciones alumno?).1 \leq 12 \\ 
    (inscripciones alumno?).2 = repite \\
    inscripciones' = inscripciones \oplus \{alumno? \mapsto ((inscripciones alumno?).1, inscripto) \} \\
    registrados' = registrados
\end{schema}

\begin{schema}{ReinscribirAlumnoNoEncontradoE}
    \Xi MisEstudiantes \\
    alumno? : ALUMNO
    \where
    alumno? \notin registrados
\end{schema}

\begin{schema}{ReinscribirAlumnoGraduadoE}
    \Xi MisEstudiantes \\
    alumno? : ALUMNO
    \where
    alumno? \in registrados \\
    ((inscripciones alumno?)).1 = 12 \\
    ((inscripciones alumno?)).2 = promueve 
\end{schema}

\begin{schema}{ReinscribirAlumnoDobleInscripE}
    \Xi MisEstudiantes \\
    alumno? : ALUMNO
    \where
    alumno? \in registrados \\
    (inscripciones alumno?).2 = inscripto
\end{schema}

\begin{zed}
    ReinscribirAlumnoE \defs ReinscribirAlumnoGraduadoE \lor ReinscribirAlumnoDobleInscripE \lor ReinscribirAlumnoNoEncontradoE
\end{zed}
\begin{zed}
    ReinscribirAlumnoOk \defs ReinscribirAlumnoPromovidoOk \lor ReinscribirAlumnoRepitenteOk
\end{zed}
\begin{zed}
    ReinscribirAlumno \defs ReinscribirAlumnoOk \lor ReinscribirAlumnoE
\end{zed}

\begin{schema}{CerrarInscripcionOk}
    \Delta MisEstudiantes \\
    alumno?: ALUMNO \\
    estado?: ESTADO
    \where
    alumno? \in registrados \\
    inscripciones' = inscripciones \oplus \{alumno? \mapsto ((inscripciones alumno?).1, estado?) \} \\
    registrados' = registrados
\end{schema}

\begin{schema}{CerrarInscripcionEstadoInvalidoE}
    \Xi MisEstudiantes \\
    estado?: ESTADO
    \where
    estado? = inscripto
\end{schema}

\begin{schema}{CerrarInscripcionAlumnoNoEncontradoE}
    \Xi MisEstudiantes \\
    alumno? : ALUMNO
    \where
    alumno? \notin registrados
\end{schema}

\begin{zed}
    CerrarInscripcionE \defs CerrarInscripcionEstadoInvalidoE \lor CerrarInscripcionAlumnoNoEncontradoE
\end{zed}
\begin{zed}
    CerrarInscripcion \defs CerrarInscripcionOk \lor CerrarInscripcionE
\end{zed}

\begin{schema}{AlumnoEsGraduadoSiOk}
    \Xi MisEstudiantes \\
    alumno? : ALUMNO \\
    rep!: REP
    \where
    alumno? \in registrados \\
    (inscripciones alumno?).1 = 12 \\
    (inscripciones alumno?).2 = promueve \\
    rep! = alumnoEsGraduado
\end{schema}

\begin{schema}{AlumnoEsGraduadoNoOk}
    \Xi MisEstudiantes \\
    alumno? : ALUMNO \\
    rep!: REP
    \where
    alumno? \in registrados \\
    (inscripciones alumno?).1 \neq 12 \lor (inscripciones alumno?).2 \neq promueve \\
    rep! = alumnoNoEsGraduado
\end{schema}

\begin{schema}{AlumnoEsGraduadoNoEncontradoE}
    \Xi MisEstudiantes \\
    alumno? : ALUMNO \\
    rep!: REP
    \where
    alumno? \notin registrados \\
    rep! = alumnoNoEncontrado
\end{schema}

\begin{zed}
    AlumnoEsGraduadoOk \defs AlumnoEsGraduadoSiOk \lor AlumnoEsGraduadoNoOk
\end{zed}
\begin{zed}
    AlumnoEsGraduado \defs AlumnoEsGraduadoOk \lor AlumnoEsGraduadoNoEncontradoE
\end{zed}


\begin{theorem}{InscribirAlumnoPI}
    InscripcionesInv \land InscribirAlumno \implies InscripcionesInv'
\end{theorem}

\begin{zproof}[InscribirAlumnoPI]
    invoke InscribirAlumno;
    split InscribirAlumnoOk;
    simplify;
    cases;
    invoke InscribirAlumnoOk;
    invoke InscripcionesInv;
    equality substitute;
    reduce;
    next;
    invoke InscribirAlumnoRegistradoE;
    invoke \Xi MisEstudiantes;
    reduce;
    next;
\end{zproof}

