\documentclass{article}
\usepackage{amsmath}
\usepackage{amssymb}
\usepackage[T1]{fontenc}
\usepackage{indentfirst}
\usepackage{mathtools}
\usepackage{xparse} 
\usepackage{fancyhdr}
\usepackage[spanish, mexico]{babel}
\usepackage[spanish]{layout}
\usepackage[utf8]{inputenc}
\usepackage[font=small,labelfont=bf]{caption}
\usepackage[a4paper,hmargin=1in, vmargin=1.4in,footskip=0.25in]{geometry}

\usepackage{czt}

\usepackage{framed}
\newcommand{\desig}[2]{\item #1 $\approx #2$}
\newenvironment{designations}
  {\begin{leftbar}
    \begin{list}{}{\setlength{\labelsep}{0cm}
                   \setlength{\labelwidth}{0cm}
                   \setlength{\listparindent}{0cm}
                   \setlength{\rightmargin}{\leftmargin}}}
  {\end{list}\end{leftbar}}


\setlength{\parskip}{8pt}

% defining command for the curly arrow
\newcommand{\curly}{\mathrel{\leadsto}}

\DeclarePairedDelimiterX{\Iintv}[1]{\llbracket}{\rrbracket}{\iintvargs{#1}}
\NewDocumentCommand{\iintvargs}{>{\SplitArgument{1}{,}}m}
{\iintvargsaux#1}
\NewDocumentCommand{\iintvargsaux}{mm} {#1\mkern1.5mu,\mkern1.5mu#2}

\makeatletter
\newcommand*{\currentname}{\@currentlabelname}
\makeatother

\addtolength{\textwidth}{0.2cm}
\setlength{\parskip}{8pt}
\setlength{\parindent}{0.5cm}
\linespread{1.5}


\renewcommand*\contentsname{\LARGE Índice}

\begin{document}

\begin{titlepage}
  \thispagestyle{empty}
  \begin{center}
    
    % Title
    {\huge \textbf{Verificación de sofware} \\[0.4cm]}
    {\large Verificación de un sistema bla bla} \\
    \noindent
    
    \vfill
    \vfill
    \vfill
    {\Large Autor: \par}
    {\Large Juan Ignacio Farizano\par}
  
    \vfill
    % Bottom of the page
    Trabajo Práctico Final - Ingeniería de Software II \\
    Departamento de Ciencias de la Computaci\'on\\
    Facultad de Ciencias Exactas, Ingenier\'ia y Agrimensura\\
    Universidad Nacional de Rosario \\
    Rosario, Santa Fe, Argentina\\[0.4cm]
    {\large \today} 
  \end{center}
  \end{titlepage}

\section*{El problema}
Queremos especificar el funcionamiento de un sistema que utilizan directivos de escuelas primarias y secundarias para inscribir y reinscribir alumnos (por ej. promoción de grado, repitente, etc) en la base de datos del Ministerio de Educación, lo descrito es similar a un sistema existente pero simplificado para mantener una dificultad razonable para este trabajo práctico.


Un directivo desea inscribir o reinscribir a un alumno en su escuela. Cada alumno cuenta con un legajo donde cada entrada una representa un paso del alumno por el sistema educativo; cada registro describe el grado en que el alumno fue inscripto y tiene tres estados posibles:

\begin{itemize}
  \item \textbf{Inscripto}: el alumno fue inscripto y es habilitado a cursar el grado registrado
  \item \textbf{Promueve}: el alumno cumplió los requisitos para promocionar el grado al cual se encuentra inscripto y es habilitado a ser inscripto en el grado siguiente o a graduarse si se encontraba en el 12avo grado
  \item \textbf{Repite}: el alumno no cumplió los requisitos para promocionar de grado y debe ser inscripto en el mismo grado
\end{itemize}

Además se desea llevar un registro de los alumnos graduados, es decir alumnos que haya promocionado 12avo grado.

Se deben especificar las siguientes operaciones:

1. \textbf{Inscribir}: un directivo inscribe por primera vez a un alumno a primer grado o a un alumno pre-existente al grado que corresponda según su último estado de inscripción

2. \textbf{Cerrar inscripción}: la inscripción actual es cerrada correspondientemente a si cumplió o no los requisitos para promocionar de grado

3. \textbf{Consultar repitencia}: se desea consultar si un alumno repitió el último grado que haya cursado

El legajo de un alumno será compartido globalmente entre todas las escuelas que utilicen el sistema, de esta forma se registra el historial completo de un alumno desde el primer grado hasta el último.
Para simplificar no diferenciamos entre primaria y secundaria, registramos desde 1er grado hasta 12avo, el último del secundario. Los requisitos para promocionar de grado se encuentran por fuera del sistema y no deben ser tenidos en cuenta.
Se quieren saber todos los estados que pasaron las inscripciones, por lo tanto cada estado de cada inscripción es inmutable, si un alumno repite se produce un nuevo registro con estado \textbf{Inscripto} en el mismo grado y si promociona se registra nuevamente con el estado \textbf{Promueve}.

\textbf{Nota:} Esto quizás es demasiado para el enunciado, mucho puede moverse a los comentarios entre los esquemas más abajo

\section*{Designaciones}

\section*{Especificación en Z}
  \begin{zed}
    [ALUMNO] \\
    GRADO == \nat \\
    ESTADO ::= inscripto | promueve | repite \\
    REP ::= alumnoEsRepitente | alumnoNoEsRepitente | alumnoNoEncontrado
  \end{zed}
  
  \begin{schema}{Escuela}
    legajos: ALUMNO \pfun \seq{GRADO \cross ESTADO} \\
    graduados: \power{ALUMNO}
  \end{schema}

  \begin{schema}{RequisitoGraduadoInv}
    Escuela
    \where
    \forall a \in graduados @ (last \; (legajos \; a?)).1 = 12 \land (last \; (legajos \; a?)).2 = promueve
  \end{schema}

  \begin{schema}{GraduadoTieneLegajoInv}
    Escuela
    \where
    graduados \subseteq \dom{legajos}
  \end{schema}

  \begin{schema}{EscuelaInicial}
    Escuela
    \where
    legajos = \emptyset \\
    graduados = \emptyset
  \end{schema}

  \begin{schema}{InscribirAlumnoNuevoOk}
    \Delta Escuela \\
    a? : ALUMNO
    \where
    a? \notin \dom{legajos} \\
    legajos' = legajos \cup \{a? \mapsto \langle (1, inscripto) \rangle\} \\
    graduados' = graduados
  \end{schema}

  \begin{schema}{InscribirAlumnoPromovidoOk}
    \Delta Escuela \\
    a? : ALUMNO
    \where
    a? \in \dom{legajos} \\
    (last \; (legajos \; a?)).2 = promueve \\
    1 \leq (last \; (legajos \; a?)).1 < 12 \\ 
    legajos' = legajos \oplus \{a? \mapsto legajos \; a? \cat \langle ((last \; (legajos \; a?)).1 + 1, inscripto) \rangle\} \\
    graduados' = graduados
  \end{schema}

  \begin{schema}{InscribirAlumnoRepitenteOk}
    \Delta Escuela \\
    a? : ALUMNO
    \where
    a? \in \dom{legajos} \\
    (last \; (legajos \; a?)).2 = repite \\
    1 \leq (last \; (legajos \; a?)).1 \leq 12 \\ 
    legajos' = legajos \oplus \{a? \mapsto legajos \; a? \cat \langle ((last \; (legajos \; a?)).1, inscripto) \rangle\} \\
    graduados' = graduados
  \end{schema}

  \begin{schema}{InscribirAlumnoGraduadoE}
    \Xi Escuela \\
    a? : ALUMNO
    \where
    a? \in \dom{legajos} \\
    (last \; (legajos \; a?)).2 = promueve \\
    (last \; (legajos \; a?)).1 = 12
  \end{schema}

  \begin{schema}{InscribirAlumnoDobleInscripE}
    \Xi Escuela \\
    a? : ALUMNO
    \where
    a? \in \dom{legajos} \\
    (last \; (legajos \; a?)).2 = inscripto
  \end{schema}

  \begin{zed}
    InscribirAlumnoE == InscribirAlumnoGraduadoE \lor InscribirAlumnoDobleInscripE \\
    InscribirAlumnoOk == InscribirAlumnoNuevoOk \lor InscribirAlumnoPromovidoOk \\ 
                         \lor InscribirAlumnoRepitenteOk \\
    InscribirAlumno == InscribirAlumnoOk \lor InscribirAlumnoE
  \end{zed}

  \begin{schema}{CerrarInscripcionNoGraduadoOk}
    \Delta Escuela \\
    a?: ALUMNO \\
    e?: ESTADO
    \where
    a? \in \dom{legajos} \\
    e? = repite \lor (e? = promueve \land (last \; (legajos \; a?)).1 < 12) \\
    (last \; (legajos \; a?)).2 = inscripto \\
    legajos' = legajos \oplus \{a? \mapsto legajos \; a? \cat \langle ((last \; (legajos \; a?)).1, e?) \rangle\}
  \end{schema}

  \begin{schema}{CerrarInscripcionGraduadoOk}
    \Delta Escuela \\
    a?: ALUMNO \\
    e?: ESTADO
    \where
    a? \in \dom{legajos} \\
    e? = promueve \\
    (last \; (legajos \; a?)).1 = 12 \\
    (last \; (legajos \; a?)).2 = inscripto \\
    legajos' = legajos \oplus \{a? \mapsto legajos \; a? \cat \langle ((last \; (legajos \; a?)).1, e?) \rangle\} \\
    graduados' = graduados \cup \{ a? \}
  \end{schema}

  \begin{schema}{CerrarInscripcionEstadoInvalidoE}
    \Xi Escuela \\
    e?: ESTADO
    \where
    e? = inscripto
  \end{schema}

  \begin{schema}{CerrarInscripcionAlumnoNoEncontradoE}
    \Xi Escuela \\
    a? : ALUMNO
    \where
    a? \notin \dom{legajos}
  \end{schema}

  \begin{zed}
    CerrarInscripcionE == CerrarInscripcionEstadoInvalidoE \lor CerrarInscripcionAlumnoNoEncontradoE \\
    CerarrInscripcionOk == CerrarInscripcionNoGraduadoOk \lor CerrarInscripcionGraduadoOk \\
    CerrarInscripcion == CerrarInscripcionOk \lor CerrarInscripcionE
  \end{zed}

  \begin{schema}{AlumnoEsRepitenteSiOk}
    \Xi Escuela \\
    rep!: REP
    \where
    a? \in \dom{legajos} \\
    (last \; (legajos \; a?)).2 = repite \\
    rep! = alumnoEsRepitente
  \end{schema}

  \begin{schema}{AlumnoEsRepitenteNoOk}
    \Xi Escuela \\
    rep!: REP
    \where
    a? \in \dom{legajos} \\
    (last \; (legajos \; a?)).2 \neq repite \\
    rep! = alumnoNoEsRepitente
  \end{schema}

  \begin{schema}{AlumnoEsRepitenteNoEncontradoE}
    \Xi Escuela \\
    rep!: REP
    \where
    a? \notin \dom{legajos} \\
    rep! = alumnoNoEncontrado
  \end{schema}

  \begin{zed}
    AlumnoEsRepitenteE == AlumnoEsRepitenteNoEncontradoE \\
    AlumnoEsRepitenteOk == AlumnoEsRepitenteSiOk \lor AlumnoEsRepitenteNoOk \\
    AlumnoEsRepitente == AlumnoEsRepitenteOk \lor AlumnoEsRepitenteE \\
  \end{zed}
\end{document}
