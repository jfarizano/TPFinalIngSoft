\documentclass{article}
\usepackage{amsmath}
\usepackage{amssymb}
\usepackage[T1]{fontenc}
\usepackage{indentfirst}
\usepackage{mathtools}
\usepackage{xparse} 
\usepackage{fancyhdr}
\usepackage[spanish, mexico]{babel}
\usepackage[spanish]{layout}
\usepackage[utf8]{inputenc}
\usepackage[font=small,labelfont=bf]{caption}
\usepackage[a4paper,hmargin=1in, vmargin=1.4in,footskip=0.25in]{geometry}

\usepackage{czt}

\usepackage{framed}
\newcommand{\desig}[2]{\item #1 $\approx #2$}
\newenvironment{designations}
  {\begin{leftbar}
    \begin{list}{}{\setlength{\labelsep}{0cm}
                   \setlength{\labelwidth}{0cm}
                   \setlength{\listparindent}{0cm}
                   \setlength{\rightmargin}{\leftmargin}}}
  {\end{list}\end{leftbar}}


\setlength{\parskip}{8pt}

% defining command for the curly arrow
\newcommand{\curly}{\mathrel{\leadsto}}

\DeclarePairedDelimiterX{\Iintv}[1]{\llbracket}{\rrbracket}{\iintvargs{#1}}
\NewDocumentCommand{\iintvargs}{>{\SplitArgument{1}{,}}m}
{\iintvargsaux#1}
\NewDocumentCommand{\iintvargsaux}{mm} {#1\mkern1.5mu,\mkern1.5mu#2}

\makeatletter
\newcommand*{\currentname}{\@currentlabelname}
\makeatother

\addtolength{\textwidth}{0.2cm}
\setlength{\parskip}{8pt}
\setlength{\parindent}{0.5cm}
\linespread{1.5}


\renewcommand*\contentsname{\LARGE Índice}

\begin{document}

\begin{titlepage}
  \thispagestyle{empty}
  \begin{center}
    
    % Title
    {\huge \textbf{Verificación de sofware} \\[0.4cm]}
    {\large Verificación de un sistema bla bla} \\
    \noindent
    
    \vfill
    \vfill
    \vfill
    {\Large Autor: \par}
    {\Large Juan Ignacio Farizano\par}
  
    \vfill
    % Bottom of the page
    Trabajo Práctico Final - Ingeniería de Software II \\
    Departamento de Ciencias de la Computaci\'on\\
    Facultad de Ciencias Exactas, Ingenier\'ia y Agrimensura\\
    Universidad Nacional de Rosario \\
    Rosario, Santa Fe, Argentina\\[0.4cm]
    {\large \today} 
  \end{center}
  \end{titlepage}

\section*{El problema}
Queremos especificar el funcionamiento del sistema \emph{Mis Estudiantes} que utilizan directivos de escuelas primarias y secundarias para inscribir y reinscribir alumnos (por ej. promoción de grado, repitente, etc) en la base de datos del Ministerio de Educación de la provincia de Buenos Aires, lo descrito es similar al sistema existente pero simplificado para mantener una dificultad razonable para este trabajo práctico.


Un directivo desea inscribir o reinscribir a un alumno en su escuela. Cada alumno puede encontrarse en una de las siguientes situaciones:

\begin{itemize}
  \item \textbf{Inscripto}: el alumno fue inscripto y es habilitado a cursar el grado registrado
  \item \textbf{Promovido}: el alumno cumplió los requisitos para promocionar el grado al cual se encuentra inscripto y es habilitado a ser inscripto en el grado siguiente o a graduarse si se encontraba en el 12avo grado
  \item \textbf{Repitente}: el alumno no cumplió los requisitos para promocionar de grado y debe ser inscripto en el mismo grado
\end{itemize}

Se deben especificar las siguientes operaciones:

\begin{enumerate}
  \item \textbf{Inscribir alumno}: se inscribe por primera vez a un alumno a primer grado
  
  \item \textbf{Reinscribir alumno}: se inscribe un alumno pre-existente al grado que corresponda según su último estado de inscripción
  
  \item \textbf{Cerrar inscripción}: la inscripción actual es cerrada correspondientemente a si cumplió o no los requisitos para promocionar de grado
  
  \item \textbf{Consultar graduado}: se desea consultar si un alumno se graduó
\end{enumerate}

Para simplificar no diferenciamos entre primaria y secundaria, registramos desde 1er grado hasta 12avo, el último del secundario. Los requisitos para promocionar de grado se encuentran por fuera del sistema y no deben ser tenidos en cuenta.

\section*{Designaciones}

\section*{Especificación en Z}
\begin{zed}
    [ALUMNO] \\
    GRADO == \nat \\
    ESTADO ::= inscripto | promueve | repite \\
    REP ::= alumnoEsRepitente | alumnoNoEsRepitente | alumnoNoEncontrado
\end{zed}

\begin{schema}{MisEstudiantes}
    legajos: ALUMNO \pfun \seq{GRADO \cross ESTADO} \\
    graduados: \power{ALUMNO}
\end{schema}

\begin{schema}{RequisitoGraduadoInv}
    MisEstudiantes
    \where
    \forall a \in graduados @ (last \; (legajos \; alumno?)).1 = 12 \land (last \; (legajos \; alumno?)).2 = promueve
\end{schema}

\begin{schema}{GraduadoTieneLegajoInv}
    MisEstudiantes
    \where
    graduados \subseteq \dom{legajos}
\end{schema}

\begin{schema}{MisEstudiantesInicial}
    MisEstudiantes
    \where
    legajos = \emptyset \\
    graduados = \emptyset
\end{schema}

\begin{schema}{InscribirAlumnoNuevoOk}
    \Delta MisEstudiantes \\
    alumno? : ALUMNO
    \where
    alumno? \notin \dom{legajos} \\
    legajos' = legajos \cup \{alumno? \mapsto \langle (1, inscripto) \rangle\} \\
    graduados' = graduados
\end{schema}

\begin{schema}{InscribirAlumnoPromovidoOk}
    \Delta MisEstudiantes \\
    alumno? : ALUMNO
    \where
    alumno? \in \dom{legajos} \\
    (last \; (legajos \; alumno?)).2 = promueve \\
    (last \; (legajos \; alumno?)).1 < 12 \\ 
    legajos' = legajos \oplus \{alumno? \mapsto legajos \; alumno? \cat \langle ((last \; (legajos \; alumno?)).1 + 1, inscripto) \rangle\} \\
    graduados' = graduados
\end{schema}

\begin{schema}{InscribirAlumnoRepitenteOk}
    \Delta MisEstudiantes \\
    alumno? : ALUMNO
    \where
    alumno? \in \dom{legajos} \\
    (last \; (legajos \; alumno?)).2 = repite \\
    (last \; (legajos \; alumno?)).1 \leq 12 \\ 
    legajos' = legajos \oplus \{alumno? \mapsto legajos \; alumno? \cat \langle ((last \; (legajos \; alumno?)).1, inscripto) \rangle\} \\
    graduados' = graduados
\end{schema}

\begin{schema}{InscribirAlumnoGraduadoE}
    \Xi MisEstudiantes \\
    alumno? : ALUMNO
    \where
    alumno? \in \dom{legajos} \\
    (last \; (legajos \; alumno?)).2 = promueve \\
    (last \; (legajos \; alumno?)).1 = 12
\end{schema}

\begin{schema}{InscribirAlumnoDobleInscripE}
    \Xi MisEstudiantes \\
    alumno? : ALUMNO
    \where
    alumno? \in \dom{legajos} \\
    (last \; (legajos \; alumno?)).2 = inscripto
\end{schema}

\begin{zed}
    InscribirAlumnoE == InscribirAlumnoGraduadoE \lor InscribirAlumnoDobleInscripE \\
    InscribirAlumnoOk == InscribirAlumnoNuevoOk \lor InscribirAlumnoPromovidoOk \\ 
                         \lor InscribirAlumnoRepitenteOk \\
    InscribirAlumno == InscribirAlumnoOk \lor InscribirAlumnoE
\end{zed}

\begin{schema}{CerrarInscripcionNoGraduadoOk}
    \Delta MisEstudiantes \\
    alumno?: ALUMNO \\
    estado?: ESTADO
    \where
    alumno? \in \dom{legajos} \\
    estado? = repite \lor (estado? = promueve \land (last \; (legajos \; alumno?)).1 < 12) \\
    (last \; (legajos \; alumno?)).2 = inscripto \\
    legajos' = legajos \oplus \{alumno? \mapsto legajos \; alumno? \cat \langle ((last \; (legajos \; alumno?)).1, estado?) \rangle\}
\end{schema}

\begin{schema}{CerrarInscripcionGraduadoOk}
    \Delta MisEstudiantes \\
    alumno?: ALUMNO \\
    estado?: ESTADO
    \where
    alumno? \in \dom{legajos} \\
    estado? = promueve \\
    (last \; (legajos \; alumno?)).1 = 12 \\
    (last \; (legajos \; alumno?)).2 = inscripto \\
    legajos' = legajos \oplus \{alumno? \mapsto legajos \; alumno? \cat \langle ((last \; (legajos \; alumno?)).1, estado?) \rangle\} \\
    graduados' = graduados \cup \{ alumno? \}
\end{schema}

\begin{schema}{CerrarInscripcionEstadoInvalidoE}
    \Xi MisEstudiantes \\
    estado?: ESTADO
    \where
    estado? = inscripto
\end{schema}

\begin{schema}{CerrarInscripcionAlumnoNoEncontradoE}
    \Xi MisEstudiantes \\
    alumno? : ALUMNO
    \where
    alumno? \notin \dom{legajos}
\end{schema}

\begin{zed}
    CerrarInscripcionE == CerrarInscripcionEstadoInvalidoE \lor CerrarInscripcionAlumnoNoEncontradoE \\
    CerarrInscripcionOk == CerrarInscripcionNoGraduadoOk \lor CerrarInscripcionGraduadoOk \\
    CerrarInscripcion == CerrarInscripcionOk \lor CerrarInscripcionE
\end{zed}

\begin{schema}{AlumnoEsRepitenteSiOk}
    \Xi MisEstudiantes \\
    rep!: REP
    \where
    alumno? \in \dom{legajos} \\
    (last \; (legajos \; alumno?)).2 = repite \\
    rep! = alumnoEsRepitente
\end{schema}

\begin{schema}{AlumnoEsRepitenteNoOk}
    \Xi MisEstudiantes \\
    rep!: REP
    \where
    alumno? \in \dom{legajos} \\
    (last \; (legajos \; alumno?)).2 \neq repite \\
    rep! = alumnoNoEsRepitente
\end{schema}

\begin{schema}{AlumnoEsRepitenteNoEncontradoE}
    \Xi MisEstudiantes \\
    rep!: REP
    \where
    alumno? \notin \dom{legajos} \\
    rep! = alumnoNoEncontrado
\end{schema}

\begin{zed}
    AlumnoEsRepitenteE == AlumnoEsRepitenteNoEncontradoE \\
    AlumnoEsRepitenteOk == AlumnoEsRepitenteSiOk \lor AlumnoEsRepitenteNoOk \\
    AlumnoEsRepitente == AlumnoEsRepitenteOk \lor AlumnoEsRepitenteE \\
\end{zed}


\section*{Setlog}

\section*{Demostraciones Z/Eves}
\begin{theorem}{InscribirAlumnoPI}
  InscripcionesInv \land InscribirAlumno \implies InscripcionesInv'
\end{theorem}

\begin{zproof}[InscribirAlumnoPI]
  invoke InscribirAlumno;
  split InscribirAlumnoOk;
  simplify;
  cases;
  invoke InscribirAlumnoOk;
  invoke InscripcionesInv;
  equality substitute;
  reduce;
  next;
  invoke InscribirAlumnoRegistradoE;
  invoke \Xi MisEstudiantes;
  reduce;
  next;
\end{zproof}
\end{document}

